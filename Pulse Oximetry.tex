\documentclass[12pt]{article}
\begin{document}
\title{A Latex report on Pulse Oximetry}
\maketitle{}

\section{Introduction}
Pulse oximetry is a noninvasive method for monitoring a person's oxygen saturation. Peripheral oxygen saturation (SpO2) readings are typically within 2 percent accuracy (within 4 pertcent accuracy in 95 percent of cases) of the more accurate (and invasive) reading of arterial oxygen saturation (SaO2) from arterial blood gas analysis.But the two are correlated well enough that the safe, convenient, noninvasive, inexpensive pulse oximetry method is valuable for measuring oxygen saturation in clinical use.

The most common approach is transmissive pulse oximetry. In this approach, a sensor device is placed on a thin part of the patient's body, usually a fingertip or earlobe, or an infant's foot. Fingertips and earlobes have higher blood flow rates than other tissues, which facilitates heat transfer.[1] The device passes two wavelengths of light through the body part to a photodetector. It measures the changing absorbance at each of the wavelengths, allowing it to determine the absorbances due to the pulsing arterial blood alone, excluding venous blood, skin, bone, muscle, fat, and (in most cases) nail polish.

Reflectance pulse oximetry is a less common alternative to transmissive pulse oximetry. This method does not require a thin section of the person's body and is therefore well suited to a universal application such as the feet, forehead, and chest, but it also has some limitations. Vasodilation and pooling of venous blood in the head due to compromised venous return to the heart can cause a combination of arterial and venous pulsations in the forehead region and lead to spurious SpO2 results. Such conditions occur while undergoing anaesthesia with endotracheal intubation and mechanical ventilation or in patients in the Trendelenburg position.


\section{Need of Pulse Oximetry}
Pulse oximetry may be used to see if there is enough oxygen in the blood. This information is needed in many kinds of situations. It may be used

1.During or after surgery or procedures that use sedation

2.To see how well lung medicines are working

3.To check a person’s ability to handle increased activity levels

4.To see if a ventilator is needed to help with breathing, or to see how well it’s working

5.To check a person has moments when breathing stops during sleep (sleep apnea)

Pulse oximetry is also used to check the health of a person with any condition that affects blood oxygen levels, such as Heart attack,Heart Failure,Anemia,Lung Cancer,Asthma,Pneumonia.

\section{History of Pulse oximetry}
In 1935, German physician Karl Matthes (1905–1962) developed the first two-wavelength ear O2 saturation meter with red and green filters (later red and infrared filters). It was the first device to measure O2 saturation.

The original oximeter was made by Glenn Allan Millikan in the 1940s. In 1943, published in 1949, Earl Wood added a pressure capsule to squeeze blood out of the ear so as to obtain an absolute O2 saturation value when blood was readmitted. The concept is similar to today's conventional pulse oximetry, but was difficult to implement because of unstable photocells and light sources; today this method is not used clinically. In 1964 Shaw assembled the first absolute reading ear oximeter, which used eight wavelengths of light.

The first pulse oximetry was developed in 1972 by Japanese bioengineers Takuo Aoyagi and Michio Kishi at Japanese medical electronic equipment manufacturer Nihon Kohden, using the ratio of red to infrared light absorption of pulsating components at the measuring site. Nihon Kohden manufactured the first pulse oximeter, Ear Oximeter OLV-5100. Surgeon Susumu Nakajima and his associates first tested the device in patients, reporting it in 1975.However, Nihon Kohden suspended the development of pulse oximetry and did not apply for a basic patent of pulse oximetry except in Japan. In 1977, Minolta commercialized the first finger pulse oximeter OXIMET MET-1471. In the U.S., it was commercialized by Biox in 1980.

By 1987, the standard of care for the administration of a general anesthetic in the U.S. included pulse oximetry. From the operating room, the use of pulse oximetry rapidly spread throughout the hospital, first to recovery rooms, and then to intensive care units. Pulse oximetry was of particular value in the neonatal unit where the patients do not thrive with inadequate oxygenation, but too much oxygen and fluctuations in oxygen concentration can lead to vision impairment or blindness from retinopathy of prematurity (ROP). Furthermore, obtaining an arterial blood gas from a neonatal patient is painful to the patient and a major cause of neonatal anemia.Motion artifact can be a significant limitation to pulse oximetry monitoring, resulting in frequent false alarms and loss of data. This is because during motion and low peripheral perfusion, many pulse oximeters cannot distinguish between pulsating arterial blood and moving venous blood, leading to underestimation of oxygen saturation. Early studies of pulse oximetry performance during subject motion made clear the vulnerabilities of conventional pulse oximetry technologies to motion artifact.

In 1995, Masimo introduced Signal Extraction Technology (SET) that could measure accurately during patient motion and low perfusion by separating the arterial signal from the venous and other signals. Since then, pulse oximetry manufacturers have developed new algorithms to reduce some false alarms during motion, such as extending averaging times or freezing values on the screen, but they do not claim to measure changing conditions during motion and low perfusion. So there are still important differences in performance of pulse oximeters during challenging conditions.Also in 1995, Masimo introduced perfusion index, quantifying the amplitude of the peripheral plethysmograph waveform. Perfusion index has been shown to help clinicians predict illness severity and early adverse respiratory outcomes in neonates,predict low superior vena cava flow in very low birth weight infants, provide an early indicator of sympathectomy after epidural anesthesia, and improve detection of critical congenital heart disease in newborns.

Published papers have compared signal extraction technology to other pulse oximetry technologies and have demonstrated consistently favorable results for signal extraction technology. Signal extraction technology pulse oximetry performance has also been shown to translate into helping clinicians improve patient outcomes. In one study, retinopathy of prematurity (eye damage) was reduced by 58 percent in very low birth weight neonates at a center using signal extraction technology, while there was no decrease in retinopathy of prematurity at another center with the same clinicians using the same protocol but with non-signal extraction technology. Other studies have shown that signal extraction technology pulse oximetry results in fewer arterial blood gas measurements, faster oxygen weaning time, lower sensor utilization, and lower length of stay. The measure-through motion and low perfusion capabilities it has also allow it to be used in previously unmonitored areas such as the general floor, where false alarms have plagued conventional pulse oximetry. As evidence of this, a landmark study was published in 2010 showing that clinicians at Dartmouth-Hitchcock Medical Center using signal extraction technology pulse oximetry on the general floor were able to decrease rapid response team activations, ICU transfers, and ICU days. In 2020, a follow-up retrospective study at the same institution showed that over ten years of using pulse oximetry with signal extraction technology, coupled with a patient surveillance system, there were zero patient deaths and no patients were harmed by opioid-induced respiratory depression while continuous monitoring was in use.

In 2007, Masimo introduced the first measurement of the pleth variability index (PVI), which multiple clinical studies have shown provides a new method for automatic, noninvasive assessment of a patient's ability to respond to fluid administration. Appropriate fluid levels are vital to reducing postoperative risks and improving patient outcomes: fluid volumes that are too low (under-hydration) or too high (over-hydration) have been shown to decrease wound healing and increase the risk of infection or cardiac complications. Recently, the National Health Service in the United Kingdom and the French Anesthesia and Critical Care Society listed PVI monitoring as part of their suggested strategies for intra-operative fluid management.

In 2011, an expert workgroup recommended newborn screening with pulse oximetry to increase the detection of critical congenital heart disease (CCHD).The CCHD workgroup cited the results of two large, prospective studies of 59,876 subjects that exclusively used signal extraction technology to increase the identification of CCHD with minimal false positives. The CCHD workgroup recommended newborn screening be performed with motion tolerant pulse oximetry that has also been validated in low perfusion conditions. In 2011, the US Secretary of Health and Human Services added pulse oximetry to the recommended uniform screening panel. Before the evidence for screening using signal extraction technology, less than 1 percent of newborns in the United States were screened. Today, The Newborn Foundation has documented near universal screening in the United States and international screening is rapidly expanding. In 2014, a third large study of 122,738 newborns that also exclusively used signal extraction technology showed similar, positive results as the first two large studies.

High-resolution pulse oximetry (HRPO) has been developed for in-home sleep apnea screening and testing in patients for whom it is impractical to perform polysomnography. It stores and records both pulse rate and SpO2 in 1 second intervals and has been shown in one study to help to detect sleep disordered breathing in surgical patients.


\section{Advantages of Pulse Oximetry}
Pulse oximetry is particularly convenient for noninvasive continuous measurement of blood oxygen saturation. In contrast, blood gas levels must otherwise be determined in a laboratory on a drawn blood sample. Pulse oximetry is useful in any setting where a patient's oxygenation is unstable, including intensive care, operating, recovery, emergency and hospital ward settings, pilots in unpressurized aircraft, for assessment of any patient's oxygenation, and determining the effectiveness of or need for supplemental oxygen. Although a pulse oximeter is used to monitor oxygenation, it cannot determine the metabolism of oxygen, or the amount of oxygen being used by a patient. For this purpose, it is necessary to also measure carbon dioxide (CO2) levels. It is possible that it can also be used to detect abnormalities in ventilation. However, the use of a pulse oximeter to detect hypoventilation is impaired with the use of supplemental oxygen, as it is only when patients breathe room air that abnormalities in respiratory function can be detected reliably with its use. Therefore, the routine administration of supplemental oxygen may be unwarranted if the patient is able to maintain adequate oxygenation in room air, since it can result in hypoventilation going undetected.

Because of their simplicity of use and the ability to provide continuous and immediate oxygen saturation values, pulse oximeters are of critical importance in emergency medicine and are also very useful for patients with respiratory or cardiac problems,especially COPD, or for diagnosis of some sleep disorders such as apnea and hypopnea. For patients with obstructive sleep apnea, pulse oximetry readings will be in the 70–90 percent range for much of the time spent attempting to sleep.

Portable battery-operated pulse oximeters are useful for pilots operating in non-pressurized aircraft above 10,000 feet (3,000 m) or 12,500 feet (3,800 m) in the U.S. where supplemental oxygen is required. Portable pulse oximeters are also useful for mountain climbers and athletes whose oxygen levels may decrease at high altitudes or with exercise. Some portable pulse oximeters employ software that charts a patient's blood oxygen and pulse, serving as a reminder to check blood oxygen levels.

Connectivity advancements have made it possible for patients to have their blood oxygen saturation continuously monitored without a cabled connection to a hospital monitor, without sacrificing the flow of patient data back to bedside monitors and centralized patient surveillance systems.

For patients with COVID-19, pulse oximetry helps with early detection of silent hypoxia, in which the patients still look and feel comfortable, but their SpO2 is perilously low.This happens to patients either in the hospital or at home. Low SpO2 may indicate severe COVID-19-related pneumonia, requiring a ventilator.


\section{Mode of Operation}
A typical pulse oximeter uses an electronic processor and a pair of small light-emitting diodes (LEDs) facing a photodiode through a translucent part of the patient's body, usually a fingertip or an earlobe. One LED is red, with wavelength of 660 nm, and the other is infrared with a wavelength of 940 nm. Absorption of light at these wavelengths differs significantly between blood loaded with oxygen and blood lacking oxygen. Oxygenated hemoglobin absorbs more infrared light and allows more red light to pass through. Deoxygenated hemoglobin allows more infrared light to pass through and absorbs more red light. The LEDs sequence through their cycle of one on, then the other, then both off about thirty times per second which allows the photodiode to respond to the red and infrared light separately and also adjust for the ambient light baseline.

The amount of light that is transmitted (in other words, that is not absorbed) is measured, and separate normalized signals are produced for each wavelength. These signals fluctuate in time because the amount of arterial blood that is present increases (literally pulses) with each heartbeat. By subtracting the minimum transmitted light from the transmitted light in each wavelength, the effects of other tissues are corrected for, generating a continuous signal for pulsatile arterial blood. The ratio of the red light measurement to the infrared light measurement is then calculated by the processor (which represents the ratio of oxygenated hemoglobin to deoxygenated hemoglobin), and this ratio is then converted to SpO2 by the processor via a lookup table based on the Beer–Lambert law.The signal separation also serves other purposes: a plethysmograph waveform (pleth wave) representing the pulsatile signal is usually displayed for a visual indication of the pulses as well as signal quality, and a numeric ratio between the pulsatile and baseline absorbance (perfusion index) can be used to evaluate perfusion.


\section{Limitations of Pulse Oximetry}

Pulse oximetry solely measures hemoglobin saturation, not ventilation and is not a complete measure of respiratory sufficiency. It is not a substitute for blood gases checked in a laboratory, because it gives no indication of base deficit, carbon dioxide levels.The metabolism of oxygen can be readily measured by monitoring expired CO2, but saturation figures give no information about blood oxygen content. Most of the oxygen in the blood is carried by hemoglobin; in severe anemia, the blood contains less hemoglobin, which despite being saturated cannot carry as much oxygen.

Because pulse oximeter devices are calibrated in healthy subjects, the accuracy is poor for critically ill patients and preterm newborns.

Erroneously low readings may be caused by hypoperfusion of the extremity being used for monitoring (often due to a limb being cold, or from vasoconstriction secondary to the use of vasopressor agents); incorrect sensor application; highly calloused skin; or movement (such as shivering), especially during hypoperfusion. To ensure accuracy, the sensor should return a steady pulse and/or pulse waveform. Pulse oximetry technologies differ in their abilities to provide accurate data during conditions of motion and low perfusion.

Obesity, hypotension (low blood pressure), and some hemoglobin variants can reduce the accuracy of the results. Some home pulse oximeters have low sampling rates which can significantly underestimate dips in blood oxygen levels. The accuracy of pulse oximetry deteriorates considerably for readings below 80 percent.

Pulse oximetry also is not a complete measure of circulatory oxygen sufficiency. If there is insufficient bloodflow or insufficient hemoglobin in the blood (anemia), tissues can suffer hypoxia despite high arterial oxygen saturation.

Since pulse oximetry measures only the percentage of bound hemoglobin, a falsely high or falsely low reading will occur when hemoglobin binds to something other than oxygen:

Hemoglobin has a higher affinity to carbon monoxide than it does to oxygen, and a high reading may occur despite the patient's actually being hypoxemic. In cases of carbon monoxide poisoning, this inaccuracy may delay the recognition of hypoxia (low cellular oxygen level).
Cyanide poisoning gives a high reading because it reduces oxygen extraction from arterial blood. In this case, the reading is not false, as arterial blood oxygen is indeed high in early cyanide poisoning.[clarification needed]
Methemoglobinemia characteristically causes pulse oximetry readings in the mid-80s.
COPD [especially chronic bronchitis] may cause false readings.
A noninvasive method that allows continuous measurement of the dyshemoglobins is the pulse CO-oximeter, which was built in 2005 by Masimo.By using additional wavelengths,it provides clinicians a way to measure the dyshemoglobins, carboxyhemoglobin, and methemoglobin along with total hemoglobin.

Research has suggested that error rates in common pulse oximeter devices may be higher for adults with dark skin color, leading to claims of encoding systemic racism in countries with multi-racial populations such as the United States. Pulse oximetry is used for the screening of sleep apnea and other types of sleep-disordered breathing which in the United States are conditions more prevalent among minorities.



\end{document}